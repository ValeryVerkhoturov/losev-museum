% !TeX encoding = UTF-8
% !TeX spellcheck = russian-aot
% !TeX program = xelatex
\documentclass[a4paper]{article}

\usepackage[english, main=russian]{babel}
\usepackage{fontspec}
\setmainfont{CMU Serif}
\usepackage{microtype}
\usepackage{authblk}
\usepackage[left={<<}, right={>>}, leftsub={„}, rightsub={“}]{dirtytalk}


\title{Дом Лосева. Отзыв}
\author{В.\,Верхотуров}
\affil{БСБО-05-20 \\ РТУ МИРЭА}

\usepackage{hyperref}
\hypersetup{pdftitle={Дом Лосева. Отзыв}, pdfauthor={В. Верхотуров}}

\begin{document}
	\maketitle
	
	Сегодня я~посетил Дом Лосева~--- русского советского философа, писателя~--- по~адресу ул. Арбат,~33, Москва. 
	
	Экскурсия началась с~отрочества, юношества А.~В.~Лосева. Он~окончил Новочеркасскую гимназию, в~1911--1915 годах учился в~Московском университете. Становление Алексея Фёдоровича прошло в~доме М.~К.~Морозовой, вдовы промышленника и~коллекционера М.~А.~Морозова, где было учреждено Религиозно-философское общество памяти В.~Соловьева. 
	
	А.~В.~Лосев занимается мифологией, пишет \say{Первое восьмикнижие}, где классифицирует мифологию на~абсолютную (абсолютное бытие) и~относительную. Абсолютная мифология~--- откровения Бога (мира), относительная~--- продукт умственной деятельности человека (может содержать преувеличение, уменьшение). В~сборник входит произведение \say{Диалектика мифа}, в~цензурированный тираж которого вставляет крамольное дополнение, за~что тираж уничтожат, автора отправят под~арест. Пропустивший книгу философа цензор тогда в~своё оправдание говорит, что А.~В.~Лосев представляет \say{оттенок философской мысли}, на~что В. Киршон отвечает: \say{За~такие оттенки надо ставить к~стенке!} Отбывал наказание Алексей Фёдорович на~строительстве Беломорско-Балтийского канала, отпущен по~инвалидности и~благодаря стараниям супруги. А.~В.~Лосева не~приняли как философа, травля его не~оканчивалась до~смерти, уступки начались только после смерти И.~В.~Сталина. Из-за инвалидности А.~В.~Лосева требовались помощники для написания научных трудов.
	
	Алексей Фёдорович участвовал в~имяславии (полемике об~имени Божьем), которое зародилось на~Афоне. Согласно этому движению, имя~--- концентрирование сущности, в~котором содержится нетварная энергия (исихазм).
	
	В 1929~г. Алексей Федорович вместе с~супругой, Валентиной Михайловной, дали монашеские обеты, приняв имена Андроника и~Афанасии.
	
	Последний написанный текст~--- речь \say{Реальность общего: Слово о~Кирилле и~Мефодии}.
	
	Экскурсия закончилась в~читальном зале Дома Лосева, в~котором собраны уцелевшие после бомбёжки во~время ВОВ книги А.~В.~Лосева.
	
\end{document}